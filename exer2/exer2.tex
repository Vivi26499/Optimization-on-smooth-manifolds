\documentclass[en, oneside]{assignment}

\ProjectInfos{Optimization on smooth manifolds}{MATH-512}{Fall, 2024}{Exercise 2}{Due date: }{Vivi}[]{24S153073}

\begin{document}

\begin{prob} \textbf{Small Stiefel Manifold}\\
    The Stiefel manifold is the set of matrices of size $n\times p$ whose columns are orthonormal. 
    In this introductory exercise, we suggest that you work out the fact that pairs of orthonormal vectors indeed form a manifold. 
    \begin{itemize}
        \item[(1)] Show that the set
        \begin{equation*}
            \mathcal{M} = \{(x, y) \in \mathbb{R}^d \times \mathbb{R}^d : x^\top x = 1, y^\top y = 1, x^\top y = 0\}  
        \end{equation*}
        is an embedded submanifold of $\mathcal{E} = \mathbb{R}^d \times \mathbb{R}^d$.
        \item[(2)] What are the tangent spaces of $\mathcal{M}$? What is the dimension of $\mathcal{M}$?
    \end{itemize}
\end{prob}

\begin{sol}
    \begin{itemize}
        \item[(1)] Let $f: \mathcal{E} \to \mathbb{R}^3$ be defined by $f(x, y) = 
        \begin{pmatrix}x^\top x - 1\\y^\top y - 1\\x^\top y\end{pmatrix}: \mathbb{R}^d \times \mathbb{R}^d \to \mathbb{R}^3$.
        Then $\mathcal{M} = f^{-1}(0)$.
        \begin{itemize}
            \item[(a)] $f$ is smooth.\\
            Since $f$ is a polynomial, it is smooth.
            \item[(b)] $f$ is a submersion.\\
            The Jacobian matrix of $f$ is
            \begin{equation*}
                Df(x, y) = J_f(x, y) = \begin{pmatrix}
                    2x^\top & 0\\
                    0 & 2y^\top\\
                    y^\top & x^\top
                \end{pmatrix} \in \mathbb{R}^{3 \times 2d}
            \end{equation*}
            $\because x^\top x = 1$ and $y^\top y = 1$\\
            $\therefore x \neq 0$ and $y \neq 0$\\
            $\therefore Df(x, y)$ has full rank.
        \end{itemize}
        $\therefore \mathcal{M}$ is an embedded submanifold of $\mathcal{E}$.
        \item[(2)] the tangent spaces of $\mathcal{M}$ at $x \in \mathcal{M}$ is
        \begin{equation*}
            T_x\mathcal{M} = \ker Df(x, y) = \{(v, w) \in \mathbb{R}^d \times \mathbb{R}^d : x^\top v = 0, y^\top w = 0, x^\top w + y^\top v = 0\}
        \end{equation*}
        $\therefore \dim{\mathcal{M}} = \dim{\mathbb{R}^d \times \mathbb{R}^d} - \rank{Df(x, y)} = 2d - 3$
    \end{itemize}
\end{sol}

\begin{prob} \textbf{Rank-1 matrices as a manifold}\\
    Let $\mathcal{M} = \mathbb{R}^{m \times n}_r$ be the set of real $m \times n$ matrices of rank $r$.
    You will show that this set is an embedded submanifold of $\mathbb{R}^{m \times n}$ when $r = 1$.
    \begin{itemize}
        \item[(1)] Show that $\mathcal{M} = \mathbb{R}^{m \times n}_1$ is an embedded submanifold of $\mathcal{E} = \mathbb{R}^{m \times n}$.
        \item[(2)] What is the dimension of $\mathcal{M}= \mathbb{R}^{m \times n}_1$?
        \item[(3)] For \( X \in \mathcal{M} = \mathbb{R}_1^{m \times n} \), 
        write \( X = \sigma u v^\top \), where \( \sigma > 0 \) and \( u \in \mathbb{R}^m \), \( v \in \mathbb{R}^n \) with \( \| u \| = \| v \| = 1 \). 
        Show that
        \begin{equation*}
        T_X \mathcal{M} = \{ a u v^\top + w v^\top + u z^\top : a \in \mathbb{R}, w \in \mathbb{R}^m, z \in \mathbb{R}^n, u^\top w = 0, v^\top z = 0 \}.
        \end{equation*}
    \end{itemize}
\end{prob}

\begin{sol}
    \begin{itemize}
        \item[(1)] We consider the set
        \begin{equation*}
            \mathcal{M} = \mathbb{R}^{m \times n}_1 = \left\{X \in \mathbb{R}^{m \times n}: \rank{X} = 1\right\} 
        \end{equation*}
        Since $\rank{X} = 1$, $X$ must have a nonzero entry, say$X_{ij}$.\\
        First,we define the neighborhoods of $X$
        \begin{equation*}
            U_{ij} = \left\{X \in \mathbb{R}^{m \times n}: X_{ij} \neq 0\right\}
        \end{equation*}
        where $i = 1, \cdots, m$ and $j = 1, \cdots, n$.\\
        For $\forall X \in U_{ij}$, we have a neighborhood $V_{ij}$ of $X$ such that
        \begin{equation*}
            \left\{V_{ij} \in \mathbb{R}^{m \times n}: \left\lVert X - V_{ij}\right\rVert \lneq \left\lvert X_{ij}\right\rvert / 2 \right\} \subseteq U_{ij}.
        \end{equation*}
        So, $U_{ij}$ is open, therefore can  be the neighborhood of $X$.\\
        Then, we need to build the local defining functions $h_{ij}: U_{ij} \to \mathbb{R}^{(m-1) \times (n-1)}$.\\
        First, we try to build $h_{11}$. Let $X \in U_{11}$, i.e., $X_{11} \neq 0$.$X$ then can be written in block form as
        \begin{equation*}
            X = \begin{pmatrix}
                X_{11} \in \mathbb{R} & X_{12} \in \mathbb{R}^{1 \times (n-1)}\\
                X_{21} \in \mathbb{R}^{(m-1)} \times 1 & X_{22} \in \mathbb{R}^{(m-1) \times (n-1)}
            \end{pmatrix}.
        \end{equation*}
        Since $x$ has rank 1, each collumn of $\begin{pmatrix}X_{12}\\X_{22}\end{pmatrix}$ 
        is a scalar multiple of the first column$\begin{pmatrix}X_{11}\\X_{21}\end{pmatrix}$.\\
        Thus, $\exists w \in \mathbb{R}^{n-1}$ such that
        \begin{equation*}
            \begin{pmatrix}X_{12}\\X_{22}\end{pmatrix} = \begin{pmatrix}X_{11}\\X_{21}\end{pmatrix} w^\top 
            = \begin{pmatrix}X_{11}w^\top\\X_{21}w^\top\end{pmatrix}.
        \end{equation*}
        Considering that $X_{11} \neq 0$, we can get from the first row of the above equation that $w^\top = X_{11}^{-1}X_{12}$, 
        and from the second row that $X_{22} = X_{21}w^\top = X_{21}X_{11}^{-1}X_{12}$.\\
        This shows that for $X \in \mathcal{M} \cap U_{11}$, $X_{22} - X_{21}X_{11}^{-1}X_{12} = 0$, which gives us a local defining function
        \begin{equation*}
            h_{11}: U_{11} \to \mathbb{R}^{(m-1) \times (n-1)}, \quad h_{11}(X) = X_{22} - X_{21}X_{11}^{-1}X_{12}.
        \end{equation*}
        Apparently, $h_{11}$ is smooth because it is a componentwise polynomial.\\
        Moreover, the equation above shows that $\mathcal{M} \cap U_{11} \subseteq h_{11}^{-1}\left(0\right)$.
        To get the reverse inclusion, we need to find $h_{11}^{-1}\left(0\right)$.For $h(X) = 0$, we have
        \begin{equation*}
            \begin{pmatrix}X_{12}\\X_{22}\end{pmatrix} = \begin{pmatrix}X_{11}\\X_{21}\end{pmatrix}X_{11}^{-1}X_{12}.
        \end{equation*}
        Therefore
        \begin{equation*}
            X = \begin{pmatrix}
                X_{11} & X_{12}\\
                X_{21} & X_{22}
            \end{pmatrix} = \begin{pmatrix}X_{11}\\X_{21}\end{pmatrix}\begin{pmatrix}1&X_{11}^{-1}X_{12}\end{pmatrix}.
        \end{equation*}
        So, $\rank{X} \leq 1$ and $\rank{X} \geq 1$ due to the fact that $X_{11} \neq 0$, 
        which implies that $\rank{X} = 1$, i.e., $X \in \mathcal{M} \cap U_{11}$.\\
        Therefore, $\mathcal{M} \cap U_{11} = h_{11}^{-1}\left(0\right)$.\\
        Then, we need to show that $\rank{Dh_{11}} = (m-1)(n-1)$, by definition we have
        \begin{equation*}
            Dh_{11}(X)[V] = \frac{d}{d t}\left[h_{11}(X + tV)\right]|_{t=0} = 
            V_{22} - X_{11}^{-1}V_{21}X_{12} - X_{11}^{-1}X_{21}V_{12} + X_{11}^{-2}V_{11}X_{21}X_{12}.
        \end{equation*}
        We can set $V_{11} = 0$ and $V_{12} = 0$ and $V_{21} = 0$, $Dh_{11}(X)[V] = V_{22}$, 
        which can be any $(m-1) \times (n-1)$ matrix.\\
        So $Dh_{11}(X)$ is a surjection for $\forall X$, i.e., $\rank{Dh_{11}} = (m-1)(n-1)$.\\
        Therefore $h_{11}$ is indeed a local defining function on $U_{11}$.\\
        Next, for $U_{ij}$, we can define $h_{ij}$ in a similar way as $h_{11}$.\\
        For $\forall X \in U_{ij}$, $P_i X Q_j \in U_{11}$,
        where $P_i$ is $m \times m$ permutation matrix exchanging the $i$-th and the first row of $X$,
        and $Q_j$ is $n \times n$ permutation matrix exchanging the $j$-th and the first column of $X$.\\
        Then we can define $h_{ij}$ as
        \begin{equation*}
            h_{ij}(X) = h_{11}(P_i X Q_j).
        \end{equation*}
        By chain rule, we have
        \begin{equation*}
            Dh_{ij}(X)[V] = Dh_{11}(P_i X Q_j)[P_i V Q_j].
        \end{equation*}
        As the map $\mathcal{M} \cap U_{ij} \to \mathcal{M} \cap U_{11}, X \to P_i X Q_j$ is
        \begin{itemize}
            \item[(a)] smooth, so $h_{ij}$ is smooth as a composition of two smooth maps;
            \item[(b)] bijective, so $h_{ij}^{-1}(0) = \mathcal{M} \cap U_{ij}$ and $\rank{Dh_{ij}} = (m-1)(n-1)$.
        \end{itemize}
        Thus, $h_{ij}$ is a local defining function on $U_{ij}$.\\
        Finally, we can show that $\mathcal{M}$ is an embedded submanifold of $\mathcal{E}= \mathbb{R}^{m \times n}$ under the atlas $\{(U_{ij}, h_{ij})\}$.
        \item[(2)] Since $h_{ij}$ maps $U_{ij}$ to $\mathbb{R}^{(m-1) \times (n-1)}$
        \begin{equation*}
            \dim{\mathcal{M}} = \dim{\mathbb{R}^{m \times n}} - \dim{\mathbb{R}^{(m-1) \times (n-1)}} = mn - (m-1)(n-1) = m + n - 1.
        \end{equation*}
        \item[(3)] Since $X$ has rank 1, using the SVD decomposition, 
        $X$ can be written as $X = \sigma u v^\top$, where $\sigma > 0$ and $u \in \mathbb{R}^m$, $v \in \mathbb{R}^n$ with $\|u\| = \|v\| = 1$.\\
        We define 3 smooth curves:
        \begin{align*}
            \sigma: \mathbb{R} & \to \mathbb{R}, \sigma(0) = \sigma, \sigma'(0) = a\\
            u: \mathbb{R} & \to \mathbb{S}^{m-1}, u(0) = u, u'(0) = w / \sigma\\
            v: \mathbb{R} & \to \mathbb{S}^{n-1} , v(0) = v, v'(0) = z / \sigma
        \end{align*}
        Then we have a smooth curve on $\mathcal{M}$:
        \begin{equation*}
            c = \sigma u v^\top, c(0) = X.
        \end{equation*}
        Differentiating $c$ at $0$, we have
        \begin{align*}
            c'(0) & = \sigma'(0)u(0)v^\top(0) + \sigma(0)u'(0)v^\top(0) + \sigma(0)u(0)v'^\top(0)\\
            & = a u v^\top + w v^\top + u z^\top.
        \end{align*}
        As $u \in \mathbb{S}^{m-1}$ and $v \in \mathbb{S}^{n-1}$, we have $u^\top w = 0$ and $v^\top z = 0$.\\
        Therefore, 
        \begin{equation*}
            \{a u v^\top + w v^\top + u z^\top: a \in \mathbb{R}, w \in \mathbb{R}^m, z \in \mathbb{R}^n, u^\top w = 0, v^\top z = 0\} \subseteq  T_X\mathcal{M}.\\
        \end{equation*}
        On the other hand, 
        \begin{equation*}
            {\{a u v^\top + w v^\top + u z^\top: a \in \mathbb{R}, w \in \mathbb{R}^m, z \in \mathbb{R}^n, u^\top w = 0, v^\top z = 0\}}
        \end{equation*}
        has dimension $m + n - 1 = \dim{\mathcal{M}}$.\\
        In conclusion, $T_X\mathcal{M} = \{a u v^\top + w v^\top + u z^\top: a \in \mathbb{R}, w \in \mathbb{R}^m, z \in \mathbb{R}^n, u^\top w = 0, v^\top z = 0\}$.
    \end{itemize}
\end{sol}

\begin{prob} \textbf{Product manifolds}\\
    Let $\mathcal{M}$ be an embedded submanifold of a linear space $\mathcal{E}$. Likewise, let $\mathcal{M}'$ be an embedded submanifold of a (possibly different) linear space $\mathcal{E}'$.
    \begin{itemize}
        \item[(1)] Show that $\mathcal{M} \times \mathcal{M}'$ is an embedded submanifold of $\mathcal{E} \times \mathcal{E}'$.
        \item[(2)] How are the tangent spaces of the product manifold related to the tangent spaces of the base manifolds?
    \end{itemize}
\end{prob}

\begin{sol}
    \begin{itemize}
        \item[(1)] Suppose that $\dim{\mathcal{M}} = m, \dim{\mathcal{E}} = p, m \leq p$ and $\dim{\mathcal{M}'} = n, \dim{\mathcal{E}'} = q, n \leq q$.
        \begin{itemize}
            \item[(a)] $m = p$ and $n = q$, i.e., $\mathcal{M}$ and $\mathcal{M}'$ are open in $\mathcal{E}$ and $\mathcal{E}'$.\\
            Clearly, $\mathcal{M} \times \mathcal{M}'$ is open in $\mathcal{E} \times \mathcal{E}'$.
            \item[(b)] $m < p$ and $n = q$, i.e., $\mathcal{M}'$ is open in $\mathcal{E}'$ 
            and there exists a local defining function $h: U \to \mathbb{R}^{p-m}$ on $\mathcal{M}$.\\
            Thus we can define a local defining function
            \begin{equation*}
                H: U \times \mathcal{M}' \to \mathbb{R}^{p-m}, H(x, x') = h(x)
            \end{equation*}
            which is smooth and satisfies $H^{-1}(0) = \mathcal{M} \times \mathcal{M}'$, $\rank{DH(x, x')} = \rank{Dh(x)} = p - m = p + q - (m + n)$.
            \item[(c)] $m < p$ and $n < q$, i.e., there exists two local defining functions $h: U \to \mathbb{R}^{p-m}$ on $\mathcal{M}$ and $h': U' \to \mathbb{R}^{q-n}$ on $\mathcal{M}'$.\\
            Then we can define a local defining function
            \begin{equation*}
                H: U \times U' \to \mathbb{R}^{p-m} \times \mathbb{R}^{q-n}, H(x, x') = (h(x), h'(x'))
            \end{equation*}
            which is clearly smooth, and
            \begin{equation*}
                H^{-1}(0) = \{(x, x') \in \mathcal{M} \times \mathcal{M}': h(x) = 0, h'(x') = 0\} = (U \cap \mathcal{M}) \times (U' \cap \mathcal{M}') = (U \times U') \cap (\mathcal{M} \times \mathcal{M}')
            \end{equation*}
            and $\rank{DH(x, x')} = \rank{Dh(x)} + \rank{Dh'(x')} = p-m + q-n = p + q - (m + n)$.
        \end{itemize}
        \item[(2)] From the definition of the tangent space, we have
        \begin{equation*}
            T_{(x, x')}\left(\mathcal{M} \times \mathcal{M}'\right) = \ker{DH(x, x')} = \ker{Dh(x)} \times \ker{Dh'(x')} = T_x\mathcal{M} \times T_{x'}\mathcal{M}'.
        \end{equation*}
    \end{itemize}
\end{sol}

\begin{prob} \textbf{The cross is not a manifold}\\
    Show that the cross $\mathcal{X} = \{x \in \mathbb{R}^2: x_1^2 = x_2^2 \}$ is not an embedded submanifold of $\mathbb{R}^2$.
\end{prob}

\begin{sol}
    Clearly, $\mathcal{X}$ is open in $\mathbb{R}^2$, then $\dim{\mathcal{X}} \in \{0, 1\}$ to ensure $\mathcal{X}$ is an embedded manifold.\\
    Focusing on the point $x = (0, 0) \in \mathcal{X}$, we can have two smooth curves
    \begin{align*}
        \gamma_1: \mathbb{R} & \to \mathcal{X}, \gamma_1(t) = (t, t)\\
        \gamma_2: \mathbb{R} & \to \mathcal{X}, \gamma_2(t) = (t, -t)
    \end{align*}
    which satisfy that $\gamma(0) = (0, 0)$. Therefore $\gamma_1'(0) = (1, 1)$ 
    and $\gamma_2'(0) = (1, -1)$ are two linearly independent vectors in $T_{(0, 0)}\mathcal{X}$.\\
    Thus, $\dim{T_{(0, 0)}\mathcal{X}} \geq 2$, which contradicts the fact that $\dim{\mathcal{X}} \in \{0, 1\}$.\\
    Therefore, $\mathcal{X}$ is not an embedded submanifold of $\mathbb{R}^2$.
\end{sol}

\begin{prob} \textbf{Differentiating the matrix inversion}\\
    We define $U \subseteq \mathbb{R}^{n \times n}$ by $U := \det^{-1}(\mathbb{R} \setminus \{0\})$, i.e., $U$ is the set of invertible matrices. 
    As $\det : \mathbb{R}^{n \times n} \to \mathbb{R}$ is continuous, we have that $U$ is open.\\
    Let $F : U \to U$ be given by $A \mapsto A^{-1}$, i.e., $F$ is the matrix inversion. 
    The map $F$ is smooth, this can be seen by the cofactor formula, which shows that every entry of $F(A)$ is a rational function.\\
    Compute the differential of $F$.
\end{prob}

\begin{sol}
    \begin{itemize}
        \item[(1)] For $A \in U, H \in \mathbb{R}^{n \times n}$, we have
        \begin{align*}
            DF(A)[H] & = \frac{d}{dt}\left[F(A + tH)\right]|_{t=0}\\
            & = \frac{d}{dt}\left[(A + tH)^{-1}\right]|_{t=0}\\
            & = \lim_{t \to 0}\frac{(A + tH)^{-1} - A^{-1}}{t}\\
            & = \lim_{t \to 0}\frac{A^{-1} + A^{-1}(t^2 H A^{-1} H - t H) A^{-1} - A^{-1}}{t}\\
            & = \lim_{t \to 0}(t A^{-1} H A^{-1} H A^{-1} - A^{-1} H A^{-1})\\
            & = - A^{-1} H A^{-1}.
        \end{align*}
        \item[(2)] Define $A(t) = A + tH, A(t) \in U$ for $t$ small enough.\\
        Then we have
        \begin{equation*}
            F \circ A(t) \cdot  A(t) = A^{-1}(t) \cdot  A(t) = I.
        \end{equation*}
        Differentiating on both sides, we have
        \begin{align*}
            0 & = \frac{d}{dt}\left[F \circ A(t) \cdot  A(t)\right]\\
            & = \frac{d}{dt}\left[F \circ A(t)\right] \cdot A(t) + F \circ A(t) \cdot  \frac{d}{dt}\left[A(t)\right]\\
            & = DF(A(t)) [A'(t)] \cdot A(t) + F(A(t)) \cdot H\\
            & = DF(A(t)) [H] \cdot A(t) + F(A(t)) \cdot H
        \end{align*}
        In particular, for $t = 0$, we have
        \begin{equation*}
            DF(A)[H] \cdot A + A^{-1} \cdot H = 0.
        \end{equation*}
        Therefore, $DF(A)[H] = - A^{-1} H A^{-1}$.
    \end{itemize}
\end{sol}
\end{document}